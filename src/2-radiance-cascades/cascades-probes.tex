\subsection{Cascades and Probes}
Because this paper focuses on nuclear applications of \acrlong{rc}, we briefly define a couple terms commonly used in computer graphics.
These are cascades and probes.
A ``probe'' in computer-graphics is synonymous to a detector in nuclear engineering.
It is a simulation structure that has a position in space and a defined set of directions where it is interested in computing and storing contributions to the radiance field.
Note, however, that probes ``point'' structures that do not have associated volume or surface area (like a detector would).
In this way, probes are analogous to nodes in finite element analysis in that they are points in space that contain more information that just their location.

Cascades are a collection of probes that are equidistant\footnote{More on this to be discussed in \cref{sec:future-work}.} from one another and contain the same angular discretizations.
In \acrlong{rc}, cascades are arranged hierarchically: the lowest cascade provides the finest spatial resolution (the most probes) but uses the coarsest angular resolution, while higher cascades contain progressively fewer probes but have the finest angular discretization.
Radiance Cascades solves the radiance field by first computing the highest-level cascade, then successively interpolating solutions downward through each cascade until the lowest level is reached, where a final interpolation produces a continuous field.
