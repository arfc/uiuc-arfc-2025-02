\subsubsection{Radiance Interval Merging}
The definition for $\psi^g\left(\mathbf{r}_0 - a\boldsymbol{\Omega},\boldsymbol{\Omega}\right)$ also defines how radiance intervals are combined to find the final solution at $\mathbf{r_0}$.
The radiance interval merging formula, written more generally than Eq.~\ref{eq:rad-contribution}, is
\begin{equation}
    \label{eq:rad-interval-merging}
    \psi^g_{i,i+1}\left(\mathbf{r}_0, \boldsymbol{\Omega}\right) = \beta^g_{i,i+1}\left(\mathbf{r}_0, \boldsymbol{\Omega}\right)\psi^g_{i+1,i+2}\left(\mathbf{r}_0,\boldsymbol{\Omega}\right)
        + \psi^g_{i,i+1}\left(\mathbf{r}_0, \boldsymbol{\Omega}\right),
\end{equation}
where the dummy index $i$ denotes the interval bounds. 
It is important to notice that two intervals can only be merged if they are directly adjacent: they must share a bound.
Finally, it should be evident that if a ray is broken into $n$ intervals, the solution to the ray is found through recursively merging the radiance intervals, collapsing towards $\mathbf{r_0}$.


