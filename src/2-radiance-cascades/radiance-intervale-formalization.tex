\subsubsection{Formalization of Radiance Intervals}
The foundational property of the radiation field that allows for \glspl{rc} is the notion of splitting incoming radiation rays into distinct intervals that can be solved for separately (and iteratively combined). 
These intervals are called `radiance intervals', by convention from Sannikov~\cite{sannikovRadianceCascadesNovel}.

To prove this property, we begin with the characteristic form of the \gls{lbe} as
\begin{equation}
    \label{eq:lbe-characteristic-form}
    \dv{u}\psi^g\left(\mathbf{r}_0 + u\boldsymbol{\Omega},\boldsymbol{\Omega}\right)
    + \Sigma_t^g\left(\mathbf{r}_0 + u\boldsymbol{\Omega}, \boldsymbol{\Omega}\right)\psi^g\left(\mathbf{r}_0 + u\boldsymbol{\Omega},\boldsymbol{\Omega}\right)
    = 
    Q^g\left(\mathbf{r}_0 + u\boldsymbol{\Omega}, \boldsymbol{\Omega}\right).
\end{equation}
Before continuing, we first perform a change of variables from $u$ to $s$, such that $u=-s$ and $\frac{d}{du}=-\frac{d}{ds}$,~\cite[p.~210]{lewisComputationalMethodsNeutron1984a}.
This is done to solve `backwards' along each characteristic line; instead of solving for the contribution at $s$ from $\mathbf{r_0}$, we solve for the contribution at $\mathbf{r_0}$ from $s$.
Then, the characteristic form using this transform is
\begin{equation}
    \label{eq:lbe-cov-char-form}
    -\dv{s}\psi^g\left(\mathbf{r}_0 - s\boldsymbol{\Omega},\boldsymbol{\Omega}\right)
    + \Sigma_t^g\left(\mathbf{r}_0 - s\boldsymbol{\Omega}, \boldsymbol{\Omega}\right)\psi^g\left(\mathbf{r}_0 - s\boldsymbol{\Omega},\boldsymbol{\Omega}\right)
    = 
    Q^g\left(\mathbf{r}_0 - s\boldsymbol{\Omega}, \boldsymbol{\Omega}\right).
\end{equation}
This is an linear ordinary differential equation, and so it can be solved for by use of an integrating factor. 
We define the integrating factor as
\begin{equation}
    I = e^{-\int_0^s \dd{s'} \Sigma^g_t\left(\mathbf{r}_0 - s'\boldsymbol{\Omega},\boldsymbol{\Omega}\right)}.
\end{equation}
Henceforth, the independent variables are dropped purely for conciseness, and everything is assumed to be a function of $\mathbf{r}_0 - s'\boldsymbol{\Omega}$ and $\boldsymbol{\Omega}$ unless otherwise specified. 
Multiplying Eq.~\ref{eq:lbe-cov-char-form} by our integrating factor, we obtain
\begin{equation}
    \label{eq:lbe-if-characteristic-form}
    -e^{-\int_0^s \dd{s'} \Sigma^g_t}\dv{s}\psi^g
    + e^{-\int_0^s \dd{s'} \Sigma^g_t}\Sigma_t^g\psi^g
    = 
    e^{-\int_0^s \dd{s'} \Sigma^g_t}Q^g.
\end{equation}
The left hand side of this equation is the output of a product rule, specifically
\begin{equation}
    \dv{s}\left[-e^{-\int_0^s \dd{s'} \Sigma^g_t}\psi^g\right]
    =-e^{-\int_0^s \dd{s'} \Sigma^g_t}\dv{s}\psi^g
    + e^{-\int_0^s \dd{s'} \Sigma^g_t}\Sigma_t^g\psi^g.
\end{equation}
Then, we insert this relation into Eq.~\ref{eq:lbe-if-characteristic-form}, and integrate both sides with respect to $s$. We set the bounds for this integration to be from $a$ to $b$, such that $0\leq a\leq b\leq\tau$ where $\tau$ is the endpoint of the ray. These bounds are chosen instead of the typical $0$ to $\tau$, because we are only interested in the incoming radiation along this interval at $s=a$.

\begin{align}
    \dv{s}\left[-e^{\int_0^s \dd{s'} \Sigma^g_t}\psi^g\right] 
        &= e^{-\int_0^s \dd{s'} \Sigma^g_t}Q^g\notag\\
    \int_a^{b}\dd{s}\dv{s}\left[-e^{-\int_0^s \dd{s'} \Sigma^g_t}\psi^g\right]
        &= \int_a^{b} \dd{s}\,e^{-\int_0^s \dd{s'} \Sigma^g_t}Q^g
\end{align}

 For brevity, we define $\psi^g(x)=\psi^g\left(\mathbf{r}_0 - x\boldsymbol{\Omega},\boldsymbol{\Omega}\right)$. Evaluating the left hand side integral, rearranging, and solving for $\psi^g(a)$, we find
\begin{align}
    e^{-\int_0^{a} \dd{s'} \Sigma^g_t}\psi^g(a)
        &= e^{-\int_0^{b} \dd{s'} \Sigma^g_t}\psi^g(b) + \int_a^{b} \dd{s}\,e^{-\int_0^s \dd{s'} \Sigma^g_t}Q^g\notag\\
    \psi^g(a) 
        &= e^{\int_0^{a} \dd{s'} \Sigma^g_t} \left[
            e^{-\int_0^{b} \dd{s'} \Sigma^g_t}\psi^g(b) + \int_a^{b} \dd{s}\,e^{-\int_0^s \dd{s'} \Sigma^g_t}Q^g
        \right]\notag\\
         &= e^{\int_0^{a} \dd{s'} \Sigma^g_t - \int_0^{b} \dd{s'} \Sigma^g_t}\psi^g(b)
            + e^{\int_0^{a} \dd{s'} \Sigma^g_t} \int_a^{b} \dd{s}\,e^{-\int_0^s \dd{s'} \Sigma^g_t}Q^g
\end{align}

Now, because the exponential in front of the integral is a function of $s'$ and not $s$, it can be factored into the integral, and thus 
\begin{align}
        \psi^g(a) &= e^{-\int_a^{b} \dd{s'} \Sigma^g_t}\psi^g(b) + \int_a^{b} \dd{s}\,e^{\int_0^a \dd{s'} \Sigma^g_t-\int_0^{s} \dd{s'} \Sigma^g_t}Q^g\notag\\
    \label{eq:rc-interval-almost}
        &= e^{-\int_a^{b} \dd{s'} \Sigma^g_t}\psi^g(b) + \int_a^{b} \dd{s}\,e^{-\int_a^s \dd{s'} \Sigma^g_t}Q^g.
\end{align}
For completeness and clarity, we now expand out the independent variables in each term.
\begin{align}
    \label{eq:rc-interval-full}
    \begin{split}
    \psi^g\left(\mathbf{r}_0 - a\boldsymbol{\Omega},\boldsymbol{\Omega}\right)
        =\, &e^{-\int_a^{b} \dd{s'} \Sigma^g_t\left(\mathbf{r}_0 - s'\boldsymbol{\Omega},\boldsymbol{\Omega}\right)}\psi^g\left(\mathbf{r}_0 - b\boldsymbol{\Omega},\boldsymbol{\Omega}\right)\\
            &+ \int_a^{b} \dd{s}e^{-\int_a^s \dd{s'} \Sigma^g_t\left(\mathbf{r}_0 - s'\boldsymbol{\Omega},\boldsymbol{\Omega}\right)}Q^g\left(\mathbf{r}_0 - s\boldsymbol{\Omega},\boldsymbol{\Omega}\right)
    \end{split}
\end{align}
Now, we formally define the transparency ($\beta^g_{a,b}$) and the radiance interval ($\psi^g_{a,b}$) as
\begin{align}
    \label{{eq:beta-and-rad-interval}}
    \beta^g_{a,b}\left(\mathbf{r}_0, \boldsymbol{\Omega}\right) &= e^{-\int_a^b \dd{s'} \Sigma^g_t\left(\mathbf{r}_0 - s'\boldsymbol{\Omega}, \boldsymbol{\Omega}\right)}\\
    \psi^g_{a,b}\left(\mathbf{r}_0, \boldsymbol{\Omega}\right)&= \int_a^{b} \dd{s}\,e^{-\int_a^s \dd{s'} \Sigma^g_t\left(\mathbf{r}_0 - s'\boldsymbol{\Omega}, \boldsymbol{\Omega}\right)}Q^g\left(\mathbf{r}_0 - s\boldsymbol{\Omega},\boldsymbol{\Omega}\right).
\end{align}
The naming `transparency' and `radiance interval' is used for clarity, as the original derivation of \glspl{rc} by Sannikov~\cite{sannikovRadianceCascadesNovel} used these terms.
Additionally, it should be noted that these definitions differ slightly from Sannikov's~\cite{sannikovRadianceCascadesNovel} in the arguments of the cross-section and source term, but are fundamentally identical. 
The notation used here is to be consistent with the standard \gls{moc} formulation within the field of computational neutron transport. 
Finally, we can insert our definitions for the transparency and radiance interval into Eq.~\ref{eq:rc-interval-full}, finding the simplified definition for the radiation contribution at $s=a$ from the interval $\left[a,b\right]$ as
\begin{equation}
    \label{eq:rad-contribution}
    \psi^g\left(\mathbf{r}_0 - a\boldsymbol{\Omega},\boldsymbol{\Omega}\right)
        = \beta^g_{a,b}\left(\mathbf{r}_0, \boldsymbol{\Omega}\right)\psi^g\left(\mathbf{r}_0 - b\boldsymbol{\Omega},\boldsymbol{\Omega}\right)
        + \psi^g_{a,b}\left(\mathbf{r}_0, \boldsymbol{\Omega}\right).
\end{equation}

