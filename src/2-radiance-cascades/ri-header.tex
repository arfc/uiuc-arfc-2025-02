\subsection{Radiance Intervals}

This section introduces and derives the concept of radiance intervals in a neutron transport framework.
The foundational property of the radiation field that allows for \glspl{rc} is the notion of splitting incoming radiation rays into distinct intervals that can be solved for separately (and iteratively combined). 
These intervals are called ``radiance intervals'', by convention from Sannikov~\cite{sannikovRadianceCascadesNovel}.
A schematic of radiance intervals and their influence of the solution on a probe can be found in \cref{fig:ri-osborne}.
\begin{figure}[!htpb]
    \centering
    \includegraphics[width=\linewidth]{figs/radiance-interval-cmo}
    \caption{Concept of radiance intervals illustrated against standard long-characteristics.
    Taken from~\cite{osborneRadianceCascadesNovel2025}
    \label{fig:ri-osborne}.
    Caption from original source:
        \textit{Comparison of long characteristics style ray-casting from a point against two different radiance intervals over the same field.
        The three coloured primitives can be considered opaque emissive sources.
        In (a) we show the conventional ray-casting approach, whilst (b) and (c) show the radiance found inside annuli described by closer and further radiance intervals (shown in grey). 
        Note that in (b) only the blue circle is found by these samples, whilst in (c) obscured components of the orange rectangle and green triangle are sampled.}
    }
\end{figure}
