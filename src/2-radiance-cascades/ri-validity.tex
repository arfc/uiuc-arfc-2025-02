\subsubsection{Validity of Radiance Interval Formulation}
To illustrate that this formulation is a valid method of breaking a ray into segments, we select an arbitrary ray broken into two segments. 
We define the interval ranges as $\left[a=0,b\right]$ and $\left[b,c=\tau\right]$.
Using equation~\ref{eq:rad-contribution} we define $\psi^g\left(\mathbf{r}_0,\boldsymbol{\Omega}\right)$ as 
\begin{align}
    \psi^g\left(\mathbf{r}_0,\boldsymbol{\Omega}\right)
        = \beta^g_{0,b}\left(\mathbf{r}_0, \boldsymbol{\Omega}\right)\psi^g\left(\mathbf{r}_0 - b\boldsymbol{\Omega},\boldsymbol{\Omega}\right)
        + \psi^g_{0,b}\left(\mathbf{r}_0, \boldsymbol{\Omega}\right).
\end{align}
Now, again using equation~\ref{eq:rad-contribution}, we define $\psi^g\left(\mathbf{r}_0 - b\boldsymbol{\Omega},\boldsymbol{\Omega}\right)$ as 
\begin{equation}
    \label{eq:rad-contribution}
    \psi^g\left(\mathbf{r}_0 - b\boldsymbol{\Omega},\boldsymbol{\Omega}\right)
        = \beta^g_{b,\tau}\left(\mathbf{r}_0, \boldsymbol{\Omega}\right)\psi^g\left(\mathbf{r}_0 - \tau\boldsymbol{\Omega},\boldsymbol{\Omega}\right)
        + \psi^g_{b,\tau}\left(\mathbf{r}_0, \boldsymbol{\Omega}\right).
\end{equation}
Inserting this definition into that of $\psi^g\left(\mathbf{r}_0,\boldsymbol{\Omega}\right)$, again dropping arguments for brevity, we find
\begin{align}
    \psi^g\left(\mathbf{r}_0,\boldsymbol{\Omega}\right) &= \beta^g_{0,b}
            \left[ \beta^g_{b,\tau}\psi^g\left(\tau\right) + \psi^g_{b,\tau} \right]
            + \psi^g_{0,b}\notag\\
        &= e^{-\int_0^b \dd{s'}\Sigma^g_t}
            \left[ e^{-\int_b^{\tau} \dd{s'}\Sigma^g_t}\psi^g\left(\tau\right) + \int_b^{\tau} \dd{s}\,e^{-\int_b^{s} \dd{s'} \Sigma^g_t}Q^g\right] +\psi^g_{b,\tau}\notag\\
        &= e^{-\int_0^{\tau} \dd{s'}\Sigma^g_t}\psi^g\left(\tau\right) + e^{-\int_0^b \dd{s'}\Sigma^g_t}\int_b^{\tau} \dd{s}\,e^{-\int_b^{s} \dd{s'} \Sigma^g_t}Q^g + \psi^g_{b,\tau}\notag\\ 
        &=e^{-\int_0^{\tau} \dd{s'}\Sigma^g_t}\psi^g\left(\tau\right) + \int_b^{\tau} \dd{s}\,e^{-\int_0^{s} \dd{s'} \Sigma^g_t}Q^g + \int_0^{b} \dd{s}\,e^{-\int_0^{s} \dd{s'} \Sigma^g_t}Q^g \notag\\
        &= e^{-\int_0^{\tau} \dd{s'}\Sigma^g_t}\psi^g\left(\tau\right) + \int_0^{\tau} \dd{s}\,e^{-\int_0^{s} \dd{s'} \Sigma^g_t}Q^g.
\end{align}

This finding is consistent with the long-characteristic solution that is the standard in \gls{moc}, and so we conclude that rays can be broken into radiance intervals as we have defined them. 


