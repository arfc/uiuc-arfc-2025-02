\subsection{Benchmarks}
Benchmark cases for \glspl{rc} must be chosen to specifically show the advantages of the method and directly test for known problems in other methods, such as ray-effects in \glspl{sn}.
To be considered successful, we will need to implement \glspl{rc} in such a way that not only converges to the correct solution, but can also do so faster than proven methods such as \glspl{sn}.
Or, alternatively, must be proven to scale well using GPU parallelism.
We propose the following benchmark cases be used for the development and testing of \glspl{rc}.

The application of \glspl{rc} to the \glspl{bte} has been largely proven by Osborne and Sannikov~\cite{osborneRadianceCascadesNovel2025}, though with a particular application to photon transport in astrophysics.
There are two main differences between this application and ours:
\begin{enumerate}
    \item Neutron transport simulations are typically focused on length scales in the centimeters to meters, not kilometers and light-years.\label{first difference}
    \item The source term is dependent on the flux in multiplying (fission) media.\label{second difference}
\end{enumerate}

\Cref{first difference} will be straightforward to show, as there are many benchmarks that allow for testing neutron streaming.
One such set, known as the Kobayashi benchmarks~\cite{kobayashi3DRADIATIONRANSPORT2000}, has become an industry standard and has been used consistently in modern codes, such as the \glspl{sn} code Denovo from Oak Ridge National Laboratory~\cite{evansDenovoNewThreeDimensional2010b}.
They look to test streaming through voids from a fixed source into a non-multiplying material.

These benchmarks consist of three geometries:
\begin{enumerate}
    \item A source surrounded by a cubic void (\cref{fig:first-kobayashi}).
    \item A source with a vertical (chimney-like) void (\cref{fig:second-kobayashi}).
    \item A source with a ``dog leg''\footnote{Yes, that is the actual name in the paper.} void duct (\cref{fig:third-kobayashi}).
\end{enumerate}
\begin{figure}
    \centering
    \begin{subfigure}{0.45\textwidth}
        \includegraphics[width=\textwidth]{figs/2D-kobayashi-1}
        \caption{2D}
    \end{subfigure}
    \hfill
    \begin{subfigure}{0.45\textwidth}
        \includegraphics[width=\textwidth]{figs/3D-kobayashi-1}
        \caption{3D}
    \end{subfigure}
    \caption{First Kobayashi benchmark (taken from~\cite{kobayashi3DRADIATIONRANSPORT2000}).}
    \label{fig:first-kobayashi}
\end{figure}

\begin{figure}
    \centering
    \begin{subfigure}{0.45\textwidth}
        \includegraphics[width=\textwidth]{figs/2D-kobayashi-2}
        \caption{2D}
    \end{subfigure}
    \hfill
    \begin{subfigure}{0.45\textwidth}
        \includegraphics[width=\textwidth]{figs/3D-kobayashi-2}
        \caption{3D}
    \end{subfigure}
    \caption{Second Kobayashi benchmark (taken from~\cite{kobayashi3DRADIATIONRANSPORT2000}).}
    \label{fig:second-kobayashi}
\end{figure}

\begin{figure}
    \centering
    \begin{subfigure}{0.45\textwidth}
        \includegraphics[width=\textwidth]{figs/2D-kobayashi-3-1}
        \caption{2D $x-y$ plane.}
    \end{subfigure}
    \begin{subfigure}{0.45\textwidth}
        \includegraphics[width=\textwidth]{figs/2D-kobayashi-3-2}
        \caption{2D $y-z$ plane.}
    \end{subfigure}
    \begin{subfigure}{0.45\textwidth}
        \includegraphics[width=\textwidth]{figs/2D-kobayashi-3-3}
        \caption{2D $x-z$ plane.}
    \end{subfigure}
    \hfill
    \begin{subfigure}{0.45\textwidth}
        \includegraphics[width=\textwidth]{figs/3D-kobayashi-3}
        \caption{3D}
    \end{subfigure}
    \caption{Third Kobayashi benchmark (taken from~\cite{kobayashi3DRADIATIONRANSPORT2000}).}
    \label{fig:third-kobayashi}
\end{figure}
The surrounding non-void region is either strictly absorbing or 50\% absorbing with 50\% scattering (of the total cross section).
Values for these cases can be found in \cref{tab:kobayashi-source-and-mat-props}.
\begin{table}[htb!]
    \centering
    \begin{tabular}{cccc|c}
    &&&\textbf{Problem 1}&\textbf{Problem 2}\\
    \toprule
    \textbf{Region}& \textbf{Source} [n/cm\textsuperscript{3}-s]& $\mathbf{\Sigma_t}$ [1/cm]&\multicolumn{2}{c}{$\mathbf{\Sigma_s}$ [1/cm]}\\
    \midrule
    1&1&0.1&0&0.05\\
    2&0&$1\times 10^{-4}$&0&$0.5\times 10^{-4}$\\
    3&0&0.1&0&0.05\\
    \bottomrule
\end{tabular}
    \caption{Source strength and material properties for Kobayashi benchmark problems.}
    \label{tab:kobayashi-source-and-mat-props}
\end{table}



The ultimate goal will be to report on \glspl{rc}'s capabilities on the 3D problems as they were originally published.
However, we will propose to start with 2D ``flattened'' versions of these problems for a proof-of-concept internally, then move on to 3D implementation.

\Cref{second difference}, however, is the more interesting and relevant problem to tackle for a proof-of-concept, and thus will make up the bulk of the work for benchmark problems.

\paragraph{Liam}
C5G7 (multiplying)

