\subsection{Radiance Cascades Implementation}

\subsubsection{Toy Problem}
Rather than jumping in to implementing \glspl{rc} in the general case, we propose to start with a simple implementation of \textit{no more than two} ``toy'' problems, where we choose very simple transport problems in 2D to implement with either a Python or C++ script.
The purpose of this is purely educational, and it will do two things:
\begin{enumerate}
    \item Create an opportunity to practice programming \glspl{rc} on a small problem before jumping to generalized code.
    \item Show early results for the method applied to neutronics (internally, though it may make a cool LinkedIn post or something\ldots).
\end{enumerate}

% OpenMoC, OpenCSG (separating from OpenMC -- would need to work with developers)
\subsubsection{Implementation Target}
There were four codes that we investigated as possible implementation points.
These four codes were OpenMC, OpenSN, DexRT, and OpenMOC.
\paragraph{OpenMC}
\paragraph{OpenSN}
\paragraph{DexRT}
\paragraph{OpenMOC}

\subsubsection{Implementation Outline}
% Long time implementation outline
% - maintenance required
% - updates planned
% - planned implementation goals (forward solve, 2D, source iteration, neutron only, multiplying and fixed source...)

