\subsubsection{Multiplying Media: C5G7 Benchmark}\label{subsubsec:c5g7}
In addition to in-scattering sources, we will also be interested in testing sourcing contributions from a multiplying region (fission).
One of the most well-known benchmarks is the C5G7 benchmark, which models a simplified, reflected, quarter-core reactor with mixed-oxide and uranium-oxide fuel.
The core is moderated with water.
As described in~\cite{BenchmarkDeterministicTransport2005}, this problem looks to challenge codes' abilities to handle spatial heterogeneity.
A general schematic of this benchmark has been given in \cref{fig:c5g7-general}, and we refer the reader to the Nuclear Energy Agency report~\cite{BenchmarkDeterministicTransport2005} for specifics.
One advantage to choosing this benchmark is the ability to expand to time-dependent testing, as development is currently underway for this extension~\cite{DeterministicTimeDependentNeutron}.
\begin{figure}[!htb]
    \centering
    \includegraphics[width=0.6\textwidth]{figs/c5g7-general}
    \caption{General schematic of the C5G7 benchmark.}
    \label{fig:c5g7-general}
\end{figure}

The implementation of C5G7 requires a mostly-mature code to implement, with the major challenges being:
\begin{enumerate}
    \item 3D implementation.
    \item Complicated geometry.
    \item Multiple materials.
    \item Varying length scales (ex. between pincells vs. from assembly to vacuum boundary).
\end{enumerate}
Therefore, we propose to first create test problems using OpenMC, Serpent, or MCNP6 with multiplying media and simple geometry.
This will allow for the continued development and refinement of source iteration methods.
Then, we will build up to more complicated geometries, finally arriving at the C5G7 benchmark, which will be reported on and published.
