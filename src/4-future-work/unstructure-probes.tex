\subsection{Unstructured Probe Placement and Improved Interpolation Strategies}
%Liam's scratch thoughts:
%\begin{itemize}
%    \item Numerous meshing algorithms exist that can place nodes within the domain of arbitrary meshes (such as \texttt{gmsh}).
%    \item The requirements of \glspl{rc} lie in the spacing between probes with regard to their angular resolution.
%    As of the writing of this, it is understood that each increase in cascade ``level'' results in a doubling of the spatial distance.
%    It should therefore be very straightforward to use a mesh generator with a user-set minimum spatial discretization for cascade 0, then the generator can place the probes for all higher cascades by doubling a global mesh parameter.
%    The parameter needs to ensure that this is the \textit{maximum} distance between probes (nodes) to ensure the penumbra criteria are satisfied.
%    \item This admittedly sounds a lot like using finte element-like basis functions to interpolate between probes (nodes), though without the complicated basis functions -- we only need to use two probes to linearly interpolation, then three probes for quadratic, etc. I wonder, then if we could use something like B-splines to get high-accuracy interpolation between notes\ldots This may be too complicated, but the point is that increasing the interpolation accuracy should be mathematically straightforward, if not computationally obnoxious.
%    \item After meshing, each probe will have to hold information about the angular flux (including all of the its directions) \emph{and the nodes which it will interpolate between}. This should be as simple as using a search algorithm to find the nearest probes\ldots Or we could just use information from FEM meshers and ignore any information about the basis functions\ldots
%\end{itemize}
%\hrule

A natural extension of this work is to explore unstructured probe placement for radiance cascades, especially in constructive solid geometry (CSG) models, and to investigate higher-accuracy interpolation schemes between probes.

Modern mesh generators (e.g., \texttt{gmsh}) already provide algorithms capable of placing nodes throughout arbitrary domains.
Because \acrlong{rc} primarily impose constraints on probe spacing relative to angular resolution, these tools could be adapted directly.
In the current formulation, each cascade increases its spatial spacing by a factor of two.
Thus, cascade~0 could be produced using a mesh with a prescribed minimum spacing, and higher cascades could be derived simply by doubling a global spacing parameter.
This value must be enforced as a \emph{maximum} spacing to preserve the penumbra criterion.

Unstructured probe locations also raise the question of accurate interpolation.
The process resembles finite-element interpolation, though without the overhead of element-level basis functions and their gradients.
Since \acrshort{rc} only require interpolation between probe values, schemes such as linear, quadratic, or even spline-based interpolation could be evaluated.
While spline-type methods may introduce additional computational cost, the mathematical framework is straightforward and could significantly improve accuracy.

Finally, each probe would need to store its angular flux information and a record of its interpolation neighbors.
These neighbors could be identified using standard nearest-neighbor search routines or by reusing adjacency information provided by FEM meshers.

Overall, these directions point toward a general framework for extending cascades to unstructured CSG geometries while improving the fidelity of interpolated fields.
There is currently interest within the Radiance Cascades community, though it is unclear what the application or progress for the methods are.
