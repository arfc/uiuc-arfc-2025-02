\subsection{Unstructured Probe Placement and Improved Interpolation}
Liam's scratch thoughts:
\begin{itemize}
    \item Numerous meshing algorithms exist that can place nodes within the domain of arbitrary meshes (such as \texttt{gmsh}).
    \item The requirements of \glspl{rc} lie in the spacing between probes with regard to their angular resolution. 
    As of the writing of this, it is understood that each increase in cascade ``level'' results in a doubling of the spatial distance.
    It should therefore be very straightforward to use a mesh generator with a user-set minimum spatial discretization for cascade 0, then the generator can place the probes for all higher cascades by doubling a global mesh parameter. 
    The parameter needs to ensure that this is the \textit{maximum} distance between probes (nodes) to ensure the penumbra criteria are satisfied.
    \item This admittedly sounds a lot like using finte element-like basis functions to interpolate between probes (nodes), though without the complicated basis functions -- we only need to use two probes to linearly interpolation, then three probes for quadratic, etc. I wonder, then if we could use something like B-splines to get high-accuracy interpolation between notes\ldots This may be too complicated, but the point is that increasing the interpolation accuracy should be mathematically straightforward, if not computationally obnoxious.
    \item After meshing, each probe will have to hold information about the angular flux (including all of the its directions) \emph{and the nodes which it will interpolate between}. This should be as simple as using a search algorithm to find the nearest probes\ldots Or we could just use information from FEM meshers and ignore any information about the basis functions\ldots
\end{itemize}

