\clearpage
\appendix
\section{Graph Representation of Merge Step}\label{appendix:merge-graph}
For the example problem presented in \cref{fig:rc-merge-graph-problem}, the graphical represenation of the merge step is presented in \cref{fig:rc-merge-graph}.
\begin{figure}[!htbp]
    \centering
    \includegraphics[width=0.5\linewidth]{figs/rc-merge-graph-problem.pdf}
    \caption{Example problem of 3 cascades, with 16 probes in $C_0$ for simplicity.
    The upper index of each probe corresponds to the cascade, and the lower index is the probes index in the cascade.}
    \label{fig:rc-merge-graph-problem}
\end{figure}
\begin{figure}[!htbp]
    \centering
    \includegraphics[width=\linewidth]{figs/rc-merge-graph.pdf}
    \caption{Example merge graph for 3 cascades, with 16 probes in $C_0$ for simplicity.
    The upper index of each probe corresponds to the cascade, and the lower index is the probes index in the cascade.}
    \label{fig:rc-merge-graph}
\end{figure}